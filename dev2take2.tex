\documentclass{article}
\usepackage[utf8]{inputenc}
\usepackage[margin=1in]{geometry}

\title{LatexIt}
\author{Auto-translator built by SHASHVAT SRIVASTAVA}
\date{November 2019}

\begin{document}
\maketitle
\tableofcontents\newpage 
 \section{2012}9/5\\
Lecture I
\begin{itemize}
\item  professor has a math PhD???
\item  other professor does chemistry ??? Wanted to study drama too
\end{itemize}
Important central idea of this class: unifying\\
principles.\\
Not facts\\
only!\\
Important stuff
\begin{itemize}
\item  Read the course website.
\end{itemize}
I - Course policies, recitation sections, etc.
\begin{itemize}
\item  Office hours, recitations start next week
\item  Fill out the mock submission survey!
\end{itemize}
\begin{itemize}
\item  MITX?
\end{itemize}
\begin{itemize}
\item 4 exams + final. (This is a change!) Also you can drop an exam.
You Do need to show up.
\end{itemize}
\begin{itemize}
\item  Problem sets cannot be late. Collaboration is fine, but not copying.
\item  Different styles of learning
\begin{itemize}
\item  Paper textbook (Life), series of videos (see MITX), online textbook (!)
\item  Choose what you want for your own learning
\end{itemize}
\end{itemize}
What is this class?
\begin{itemize}
\item  Medicine rapidly developing in the present
\end{itemize}
\begin{itemize}
\item  Cancer therapy, etc.
\item  DNA fingerprinting - then privacy issues, ancestry etc.
\item  Constantly changing \& driving forward? Things like genome editing!
\end{itemize}
"Diversity of life" - what's the common operating system?\\
Different "levels" - biospheres, ecosystems, organisms, organs, tissues, cells,\\
organelles, molecules (big to small)\\
We focus on the small things because they are common.
\begin{itemize}
\item  Evolution - we won't talk much about it. 60 secondsi
\end{itemize}
– 4.5 billion years – Earth forms\\
0.5 billion yr - multicellular\\
4 billion yr - cools.\\
23.7 billion\\
0.005 billion yr - human-like\\
V 2200 21.5 billion - first nucleated cells I\\
lineage\\
renkamotes)\\
0.001 billion – Homo sapiens\\
0.0000002 billion – MIT founded.\\
organisms!\\
300\\
rms, somehow\\
(prokaryotes)\\
cell biology - covered a bit.
\begin{itemize}
\item  Important prokaryotes \textasciitilde1-2 pm us enkanpotes \textasciitilde40 pm.
\item  There is no "the cell, but it doesn't matter that much.
\begin{itemize}
\item  General abstract principles! f unt to
\end{itemize}
\end{itemize}
Organization of class
\begin{itemize}
\item  Biological functions. - biochemists (components w/o organism)
vs geneticists (organisms who components)
\end{itemize}
results: look at proteins and genes, respectively.\\
L Then in the 20th century: DNA molecular biology\\
Idea: RNA from DNA is sent off to a ribosome and\\
creates proteins!
\begin{itemize}
\item > New scientists: recombinant DNA
\item  Then came genomics: study the entire human genome
\item  Cost of human genome: \$3 billion – \$500
\end{itemize}
Basically, we are always doing more and moving forward\\
Pertubation, not observation
\begin{itemize}
\item  Data Scientists
\item  Continual improvement!
\end{itemize}
\newpage 
 \section{7012.}Lecture 2\\
9/7\\
\begin{itemize}
\item  Change recitations now!
\end{itemize}
\begin{itemize}
\item  P-set posted on Monday
\item  Fundamentals of life =
\end{itemize}
(see instructions on Stellar).\\
fundamentals of matter.\\
Periodic Table\\
Important to life: H, C, O, N, SP\\
Also Se, Si, I
\begin{itemize}
\item  Cations: Kt, Nat Ca²+ Mg 2+
\item  Anions: Cro atia
\end{itemize}
od-orbital ("d-block") atoms: Fe, Mn, Co, Ni, Cu, Zn, etc.
\begin{itemize}
\item > \textasciitilde 3 all proteins are metalloproteins
Good for catalyzing reactions, etc.
\end{itemize}
Chemical bonding
\begin{itemize}
\item  Goal is to have a complete valence shell of le-(electrons)
\begin{itemize}
\item  ''Noble gas" configuration
\end{itemize}
\end{itemize}
Who gets the (e-)? Depends on electronegativity! (x)
\begin{itemize}
\item  ability of atoms to attract (e-) from other atoms
\item  On a scale of NO-4. Low near bottom right, high near Fluorine
\end{itemize}
Na
CU\\
Since Xo Kwa\\
(1 valid - 17 vali)\\
al takes" an electron\\
transfer of election\\
from No.\\
= ionic bonds this\\
gives electrostatic attraction.\\
et Need a lot of electrons\\
(vali)\\
(4 val.)
\begin{itemize}
\item  Sharing!
\end{itemize}
IC\\
I-o-It\\
\&\\
Methane:\\
H-C-H.\\
Carbon likes to have 4 bonds.\\
Covalent bonds pairs are\\
shared\\
Nonpolar\\
xequal sharing\\
Polar
\begin{itemize}
\item  unequal sharing
\end{itemize}
Since X\\
= 2.2\\
X=2.55\\
(critical value)\\
the difference 0.35 < 0.4\\
so nonpolar.\\
H.\\
. (1 val.)\\
(6 val.)\\
lone pairs of (e-)\\
Hº H\\
Xx=2.2, Xo = 3.44 → polar.\\
H₂O = water is\\
polar molecule!\\
(important)\\
:0= C= O. CO₂ = carbon dioxide has double bonds.\\
x=2.55 X=3.44\\
the molecule s\\
Bonds are polar, but by symmetry, it is nonpolar.\\
Molecules of Life
\begin{itemize}
\item  Peptides / proteins are made of amino acids.
\begin{itemize}
\item Ca - Coocarboxylic acid. This is an example of a
\end{itemize}
\end{itemize}
Kside chain)\\
functional group.\\
alpha carbon\\
NH3\\
amino\\
L-amino acids\\
versus\\
D-amino acids (mirror image\\
Isomers\\
(same composition but\\
can't be superimposed)\\
into\\
the\\
page\\
NH3\\
cool\\
Coo-\\
NH\\
amino\\
acids together.\\
– Peptide bonds attach
\begin{itemize}
\item  Carbohydrates (sugars)
Ex Ribose
\end{itemize}
HO-CH2 ,\\
CH CÚ\\
OH\\
OH\\
assume\\
HHC\\
CHI\\
CH\\
LOH\\
OH\\
OH\\
OH\\
@\\
Deoxyribose (missing OH at 2).\\
OH = 0,\\
OH\\
Ex\\
Glucose (C6H12O6)\\
and form\\
oligosaccharides:\\
Monosaccharides (single sugars) join\\
e. g. starch, cellulose, etc.\\
Glycosidic linkage between sugars.\\
\begin{itemize}
\item 
\end{itemize}
\begin{itemize}
\item  Nucleic acids
> Ribonucleotides form RNA; Deoxyribonucleotides form DNA,
\end{itemize}
1, 2 or 3.\\
Pop - O\\
Base (AUGL.\\
for C\\
Same as RNA,\\
but U becomes\\
and missing 2' OH\\
to\\
OH\\
OH\\
phosphate,\\
NOT\\
phosphones\\
ribonucleoside\\
TP-0\\
Ex) : ATP\\
NIP\\
(forms RNA\\
vs\\
VS\\
DATP\\
ANTP\\
(forms DNA)\\
Phosphodiester linkage.\\
70 2\\
Lecture 3
\begin{itemize}
\item RNA has extra hydroxyl group over DNA
\item  Mono-, di-, or tri-phosphate
\end{itemize}
(Ex) If base is A (adenine), the nucleic acid.\\
adenosine triphosphate = ATP for RNA, and\\
deoxy adenosine " = dATP\\
\begin{itemize}
\item  In general, call it NTP where N is the RNA base (AU, G, C)
or dNTP where Nis A, I, G, or C. (ONA bases)
\end{itemize}
\begin{itemize}
\item  NDP is diphosphate, etc.
ONA stores infos RNA does more stuff
\end{itemize}
more stable\\
catalytic / reactive\\
Nucleic acids linked at 5 and 3 carbons by phosphodiester se\\
\begin{itemize}
\item  Lipids (aka fats)
Ex triglycerides
\end{itemize}
H₂C -\\
C - CH₂\\
\begin{itemize}
\item 
\end{itemize}
glycerol\\
group\\
OH\\
\begin{itemize}
\item  J
\end{itemize}
I\\
OH\\
OH\\
of these\\
fatty acids of\\
CH3 =\\
a lot of these\\
e\\
Two kinds\\
HC- CH - CH2\\
polar\\
Unsaturated\\
N has 21\\
double bond\\
O=c\\
(CH2) n (CH2)n (cheln\\
H-C=CH\\
Et\\
\begin{itemize}
\item nonpolar's
\end{itemize}
called a hydrocarbon\\
Saturated\\
has none.\\
СН3\\
CH3\\
CHz\\
Sun\\
Enz\\
carbon\\
chaine\\
Types of Lipids
\begin{itemize}
\item  Phospholipids.
\end{itemize}
phosphate\\
choline\\
(tx Phosphotidylcholine:\\
roc\\
P.\\
AN+\\
(recall Care\\
negatively positively\\
not marked)\\
charged\\
charged\\
honpolar\\
something else! polaru.\\
this is called amphipathic..\\
Polar wants to be near water, nonpolar does not.\\
hydrophilic head group all\\
water (outside)\\
I\\
hydrophobic\\
tail.\\
philic\\
03\\
water\\
(in cell)\\
11 hydrophobic\\
hydrocalled the phospholipid bilayer.\\
This process is driven entropically.\\
(thermodynamically favorable)
\begin{itemize}
\item Steroids
\end{itemize}
. Vitamins\\
EX) Vitamin A, D, E,k fat soluble (more nonpolar bonds{]}\\
Vitamin B9 water soluble\\
Interactions that hold molecules together from strong to weak
\begin{itemize}
\item  Covalent bonds (C-C, o H, etc.) (80-110 kcal/mold.
\end{itemize}
\begin{itemize}
\item  Electrostatic interactions (NH3 - Coo-) (3-8 kcal/mol) Formal charges)
Note: Nat C1 ionic bonds are much stronger! > 100 kcal/mol
\item  Hydrogen bonds Care not bonds{]}
\end{itemize}
stuff - (= ....\\
(3-7 kcal/mol)\\
7 \textasciitilde47 HO- stuff\\
kcal/mol vs 110 kcal/mol\\
some\\
Defn): Interaction between Hina polar bond and\\
electronegative atom wl lone pair of (e).\\
\begin{itemize}
\item 
\end{itemize}
Hydrophobic interactions\\
(between nonpolar substances) in the presence of water\\
③ – CH3 .... CH₂ -\\
(\textasciitilde1-2 kcal/mol)\\
van der Waals interactions -\\
transient\\
polarization.\\
(8 5) -\\
(85)\\
(\textasciitilde0.1-1 kcal/mol,\\
distance-dependent)\\
There are also unfavorable interactions:
\begin{itemize}
\item  Electrostatic repulsions
\item  Repulsion due to "buried charge in the middle
\end{itemize}
(Ex) Protein structure
\begin{itemize}
\item  Amino acids can be polar, nonpolar, t or – charged too.
Some dependent on pka the pH where acid is 50\% protonated.
\end{itemize}
(pH=pka - log\\
( {[}protonated ?\\
{[}deprotonated{]}\\
Asp\\
Aspartic acid.\\
OH\\
pka=3.7 so if pH <317 more protonated and vice versa.\\
Asp\\
or\\
So at neutral pH \textasciitilde7,\\
pero\\
this is dominant species.\\
Bond formed\\
Macromolecule\\
I\\
made of\\
Proteins\\
amino acids\\
peptide\\
through dehydration\\
synthesis\\
Carbohydrates\\
monosaccharides\\
glycosidic\\
(loss of\\
water)\\
Nucleic\\
acids\\
nucleotides\\
phosphodiester\\
Lipids\\
fatty\\
acids\\
ester\\
\newpage 
 \section{7012}9/12\\
Lecture 4\\
Different types of structure\\
1 (primary) = Sequence of amino acids\\
connected by peptide\\
20 (secondary) = x-helix or B-strand\\
(most proteins have many)\\
(800 vs m\\
i es\\
hydrogen bonds here too\\
very stable,\\
hydrogen bonds\\
between\\
backbone\\
J=0... I\\
G H 0 H\\
Ho Ho\\
atoms\\
\begin{itemize}
\item > parallel or antiparallel
\end{itemize}
30 (tertiary) = protein fold (the 3D arrangement of the secondary structures)
\begin{itemize}
\item  a-helix referred to (ribbon drawing
\item -ß-strands shown as or
\end{itemize}
Ex\\
chain w/ two domains\\
(self-folding section)\\
B-barrel\\
domain\\
8\\
I\\
"a-helical\\
domain"\\
4° (quaternary) = essentially number lanangements of chans.\\
1 chain = monomer, 2 chains = dimer, 3 chains a trimer,\\
4 chains = tetramer. "homo" = same "hetero" = different\\
amma adds,\\
S -\\
S\\
factors that stabilize protein structure\\
covalent bonds — 1 (peptide bond), 30 (disulfide bonds) 4°\\
electrostatic interactions -3° (side chains, e.g. Glu, Arg), 4°\\
hidrogen bonds 2 (from backbone atoms), 3° (some backbone\\
hydrophobie interaction → 30, 4°\\
van der Waals - 30, 4°\\
e backbone interaction\\
Why do proteins fold?
\begin{itemize}
\item  Hydrophobic effect (entropically driven)
y hydrophobic, nonpolar atoms inside hydrophilic, polar atoms
\end{itemize}
Chemical reactions of life\\
C6H12Oo {[}has lots of stored energy{]} + 6 O2 →6 CO₂ + 6 H₂O\\
glucose)\\
+ energy\\
→ Where does energy come from? Need high availability (plants),\\
favorable reaction (exergonic 3 vs enderganic\\
{[}spontaneous\\
{[}nonspontaneous{]}\\
AGO\\
CAGO > 0\\
Standard state
\begin{itemize}
\item > 46° = difference in Gibbs free energy a
\item  For cellular resp., OG = -673 kcals/moll
\end{itemize}
But rate of reaction (kinetics) is a factor too. So it is pretty\\
slow unless you use enzymes to catalyze reaction and speed it up\\
wlo catalyst\\
nowl catalyst\\
free\\
free\\
energy reactants\\
Jactivata\\
enery\\
transition)\\
state\\
energy\\
actisation\\
enes\\
products\\
reaction\\
coordinate\\
reachon coordinate\\
\end{document}